\documentclass[12pt]{article}
%%---------------------------------------------------------------------
% packages
% geometry
\usepackage{geometry}
% font
\usepackage{fontspec}
\defaultfontfeatures{Mapping=tex-text}  %%如果没有它,会有一些 tex 特殊字符无法正常使用,比如连字符。
\usepackage{xunicode,xltxtra}
\usepackage[BoldFont,SlantFont,CJKnumber,CJKchecksingle]{xeCJK}  % \CJKnumber{12345}: 一万二千三百四十五
\usepackage{CJKfntef}  %%实现对汉字加点、下划线等。
\usepackage{pifont}  % \ding{}
% math
\usepackage{amsmath,amsfonts,amssymb}
% color
\usepackage{color}
\usepackage{listings}
\usepackage{xcolor}
\definecolor{EYE}{RGB}{199,237,204}
\definecolor{FLY}{RGB}{128,0,128}
\definecolor{ZHY}{RGB}{139,0,255}
% graphics
\usepackage[americaninductors,europeanresistors]{circuitikz}
\usepackage{tikz}
\usetikzlibrary{positioning,arrows,shadows,shapes,calc,mindmap,trees,backgrounds}  % placements=positioning
\usepackage{graphicx}  % \includegraphics[]{}
\usepackage{subfigure}  %%图形或表格并排排列
% table
\usepackage{colortbl,dcolumn}  %% 彩色表格
\usepackage{multirow}
\usepackage{multicol}
\usepackage{booktabs}
% code
\usepackage{fancyvrb}
\usepackage{listings}%代码高亮
\lstset{language=C++}%这条命令可以让LaTeX排版时将C++键字突出显示
\lstset{breaklines}%这条命令可以让LaTeX自动将长的代码行换行排版
\lstset{extendedchars=false}
% title
\usepackage{titlesec}
% head/foot
\usepackage{fancyhdr}
% ref
\usepackage{hyperref}
% pagecolor
\usepackage[pagecolor={EYE}]{pagecolor}
% tightly-packed lists
\usepackage{mdwlist}

\usepackage{styles/iplouccfg}
\usepackage{styles/zhfontcfg}
\usepackage{styles/iplouclistings}



%%---------------------------------------------------------------------
% settings
% geometry
\geometry{left=2cm,right=1cm,top=2cm,bottom=2cm}  %设置 上、左、下、右 页边距
\linespread{1.5} %行间距
% font
\setCJKmainfont{Adobe Kaiti Std}
%\setmainfont[BoldFont=Adobe Garamond Pro Bold]{Apple Garamond}  % 英文字体
%\setmainfont[BoldFont=Adobe Garamond Pro Bold,SmallCapsFont=Apple Garamond,SmallCapsFeatures={Scale=0.7}]{Apple Garamond}  %%苹果字体没有SmallCaps
\setCJKmonofont{Adobe Fangsong Std}
% graphics
\graphicspath{{figures/}}
\tikzset{
    % Define standard arrow tip
    >=stealth',
    % Define style for boxes
    punkt/.style={
           rectangle,
           rounded corners,
           draw=black, very thick,
           text width=6.5em,
           minimum height=2em,
           text centered},
    % Define arrow style
    pil/.style={
           ->,
           thick,
           shorten <=2pt,
           shorten >=2pt,},
    % Define style for FlyZhyBall
    FlyZhyBall/.style={
      circle,
      minimum size=6mm,
      inner sep=0.5pt,
      ball color=red!50!blue,
      text=white,},
    % Define style for FlyZhyRectangle
    FlyZhyRectangle/.style={
      rectangle,
      rounded corners,
      minimum size=6mm,
      ball color=red!50!blue,
      text=white,},
    % Define style for zhyfly
    zhyfly/.style={
      rectangle,
      rounded corners,
      minimum size=6mm,
      ball color=red!25!blue,
      text=white,},
    % Define style for new rectangle
    nrectangle/.style={
      rectangle,
      draw=#1!50,
      fill=#1!20,
      minimum size=5mm,
      inner sep=0.1pt,}
}
\ctikzset{
  bipoles/length=.8cm
}
% code
\lstnewenvironment{VHDLcode}[1][]{%
  \lstset{
    basicstyle=\footnotesize\ttfamily\color{black},%
    columns=flexible,%
    framexleftmargin=.7mm,frame=shadowbox,%
    rulesepcolor=\color{blue},%
%    frame=single,%
    backgroundcolor=\color{yellow!20},%
    xleftmargin=1.2\fboxsep,%
    xrightmargin=.7\fboxsep,%
    numbers=left,numberstyle=\tiny\color{blue},%
    numberblanklines=false,numbersep=7pt,%
    language=VHDL%
    }\lstset{#1}}{}
\lstnewenvironment{VHDLmiddle}[1][]{%
  \lstset{
    basicstyle=\scriptsize\ttfamily\color{black},%
    columns=flexible,%
    framexleftmargin=.7mm,frame=shadowbox,%
    rulesepcolor=\color{blue},%
%    frame=single,%
    backgroundcolor=\color{yellow!20},%
    xleftmargin=1.2\fboxsep,%
    xrightmargin=.7\fboxsep,%
    numbers=left,numberstyle=\tiny\color{blue},%
    numberblanklines=false,numbersep=7pt,%
    language=VHDL%
    }\lstset{#1}}{}
\lstnewenvironment{VHDLsmall}[1][]{%
  \lstset{
    basicstyle=\tiny\ttfamily\color{black},%
    columns=flexible,%
    framexleftmargin=.7mm,frame=shadowbox,%
    rulesepcolor=\color{blue},%
%    frame=single,%
    backgroundcolor=\color{yellow!20},%
    xleftmargin=1.2\fboxsep,%
    xrightmargin=.7\fboxsep,%
    numbers=left,numberstyle=\tiny\color{blue},%
    numberblanklines=false,numbersep=7pt,%
    language=VHDL%
    }\lstset{#1}}{}
% pdf
\hypersetup{pdfpagemode=FullScreen,%
            pdfauthor={Haiyong Zheng},%
            pdftitle={Title},%
            CJKbookmarks=true,%
            bookmarksnumbered=true,%
            bookmarksopen=false,%
            plainpages=false,%
            colorlinks=true,%
            citecolor=green,%
            filecolor=magenta,%
            linkcolor=cyan,%red(default)
            urlcolor=cyan}
% section
%http://tex.stackexchange.com/questions/34288/how-to-place-a-shaded-box-around-a-section-label-and-name
\newcommand\titlebar{%
\tikz[baseline,trim left=3.1cm,trim right=3cm] {
    \fill [cyan!25] (2.5cm,-1ex) rectangle (\textwidth+3.1cm,2.5ex);
    \node [
        fill=cyan!60!white,
        anchor= base east,
        rounded rectangle,
        minimum height=3.5ex] at (3cm,0) {
        \textbf{\thesection.}
    };
}%
}
\titleformat{\section}{\Large\bf\color{blue}}{\titlebar}{0.1cm}{}
% head/foot
\setlength{\headheight}{15pt}
\pagestyle{fancy}
\fancyhf{}

\chead{\color{black!50!green}Shape Matching Review}

\lfoot{\color{blue!50!green}Wang Ruchen}
\cfoot{\color{blue!50!green}\href{http://vision.ouc.edu.cn/~zhenghaiyong}{CVBIOUC}}
\rfoot{\color{blue!50!green}$\cdot$\ \thepage\ $\cdot$}
\renewcommand{\headrulewidth}{0.4pt}
\renewcommand{\footrulewidth}{0.4pt}

%%---------------------------------------------------------------------
\begin{document}
%%---------------------------------------------------------------------
%%---------------------------------------------------------------------
% \titlepage
\title{\vspace{-2em}OpenCV\vspace{-0.7em}}
\author{Wang Ruchen}
%\date{\vspace{-0.7em}December 26th, 2014\vspace{-0.7em}}
%%---------------------------------------------------------------------
\maketitle
\thispagestyle{fancy}
%%---------------------------------------------------------------------
\maketitle
\tableofcontents 

\section{安装}
\begin{enumerate}
\item 进入解压后的opencv3.0目录
    \begin{lstlisting}[language=C++]
    mkdir build
    \end{lstlisting}
\item 进入build目录
    \begin{lstlisting}[language=C++]
    cd build
    cmake -D CMAKE_BUILD_TYPE=Release -D CMAKE_INSTALL_PREFIX=/usr/local ..
    make
    sudo make install
    \end{lstlisting}
\end{enumerate}

\section{core}
    核心功能模块,包含如下内容:
\begin{itemize}
\item OpenCV基本数据结构
\item 动态数据结构
\item 绘图函数
\item 数组操作相关函数
\item 辅助功能与系统函数和宏
\item 与OpenGL的互操作
\end{itemize}
\subsection{基本数据类型}

\subsubsection{Point\_}
\begin{itemize}
\item Point\_:定义二维坐标x,y
	\begin{itemize}
	\item Point \quad Point2d(Point\_<double>) \quad Point2f(Point\_<float>) \quad Point2i(Point\_<int>)\\
	e.g.:Point2i a(x,y);
	\end{itemize}
\end{itemize}
	
\subsubsection{Point3\_}
\begin{itemize}
\item Point3\_:定义三维坐标x,y,z
\end{itemize}

\subsubsection{Size\_}
\begin{itemize}
\item Size\_:定义图片或矩阵的大小,包含width和height两个元素。
\end{itemize}

\subsubsection{Rect\_}
\begin{itemize}
\item Rect\_:定义二维图片或矩阵,包含元素x,y,width,height。x,y为起始坐标,width,height为矩形的长宽。
\end{itemize}

\subsubsection{RotatedRect}
\begin{itemize}
\item RotatedRect:旋转矩形,包含元素center,size,angle,box。


\begin{lstlisting}[language=C++]
RotatedRect(const Point2f& center, const Size2f& size, float angle)
\end{lstlisting}
		\begin{itemize}
		\item center:矩形的中心坐标
		\item size:矩形的长宽
		\item angle:顺时针旋转的角度
		\end{itemize}
\begin{lstlisting}[language=C++]
RotatedRect(const Point2f& point1, const Point2f& point2, const Point2f& point3)
\end{lstlisting}
		\begin{itemize}
		\item point1,2,3 \quad 旋转后三点的坐标,按照顺时针或者逆时针的方向给出。
		\end{itemize}
\begin{lstlisting}[language=C++]
points(Point2f pts[ ])
\end{lstlisting}
		\begin{itemize}
		\item 点数组来存储矩形顶点
		\end{itemize}
\begin{lstlisting}[language=C++]
 boundingRect()
 \end{lstlisting}
 计算点集的最外面矩形边界

\end{itemize}

\subsubsection{Matx}
定义大小、数据类型已知的矩阵
\begin{itemize}
\item Matx12f,Matx12d$\cdots$
\end{itemize}

\subsubsection{Vec}
短向量,基于Matx。
\begin{itemize}
\item Vec2b \quad Vec<uchar,2>
\item Vec2s \quad Vec<short,2>
$\cdots$
\item Vec2f,Vec2i,Vec2b
\end{itemize}

\subsubsection{Range}
范围
\begin{itemize}
\item 包括两个元素start,end
\end{itemize}

\subsubsection{Mat}
矩阵结构
\begin{itemize}
\item M.data数据区域的指针
\item M.dims矩阵维度
\item M.sizes维度
\item M.elemSize()每个元素占的字节空间大小,与元素类型相关,如CV\_8U
\item M.step[]用来计算元素地址,M.step[i] 表示所有比i大的维度所占空间大小
\end{itemize}
创建特殊矩阵
\begin{itemize}
\item diag提取指定矩阵的对角线
	\begin{lstlisting}[language=c++]
	Mat::diag(const cv::Mat &d)
	\end{lstlisting}
\item ones
	\begin{lstlisting}[language=C++]
	Mat::ones(int rows, int cols, int type)
	\end{lstlisting}
\item zeros 
	\begin{lstlisting}[language=C++]
	Mat::zeros(int rows, int cols, int type)
	\end{lstlisting}
\item eye 
	\begin{lstlisting}[language=C++]
	Mat::eye(int rows, int cols, int type)
	\end{lstlisting}
\end{itemize}
相关属性
\begin{itemize}
\item rows行数 \quad row(i)
\item cols列数 \quad col(i)
\item begin返回矩阵迭代器,并将其设置为第一矩阵元
\item end返回矩阵迭代器,并将其设置为最后元素之后的矩阵元
\item at返回对指定数组元素的引用
\item size替代矩阵大小规格Size(cols, rows)的方法
	\begin{lstlisting}[language=c++]
	Mat::size() const
	\end{lstlisting}
\item depth返回一个矩阵元素的深度
\item type矩阵的类型
	\begin{lstlisting}[language=c++]
	Mat::type() const
	\end{lstlisting}
\item elemSize返回矩阵元素大小(以字节为单位)
	\begin{lstlisting}[language=c++]
	Mat::elemSize() const
	\end{lstlisting}
\item total返回数组的总数
	\begin{lstlisting}[language=c++]
	Mat::total() const
	\end{lstlisting}
\end{itemize}
矩阵操作
\begin{itemize}
\item t转置矩阵
	\begin{lstlisting}[language=c++]
	Mat::t() const
	\end{lstlisting}
\item inv反转矩阵
	\begin{lstlisting}[language=c++]
	Mat::inv(int method=DECOMP_LU) const
	\end{lstlisting}
\item mul执行两个矩阵按元素相乘或这两个矩阵的除法
	\begin{lstlisting}[language=c++]
	Mat::mul(InputArray m, double scale=1) const
	\end{lstlisting}
\item cross计算3元素向量的一个叉乘积
\item dot计算两向量的点乘
	\begin{lstlisting}[language=c++]
	Mat::dot(InputArray m) const
	\end{lstlisting}
\item reshape改变矩阵的通道数量,比如把二维矩阵改变成一维矩阵
	\begin{lstlisting}[language=c++]
	Mat::reshape(int cn)
	\end{lstlisting}
\item resize更改矩阵的行数
	\begin{lstlisting}[language=c++]
	Mat::resize(size_t sz, const Scalar& s)
	\end{lstlisting}
		\begin{itemize}
		\item sz \quad 新的行数
		\item s \quad 分配给新添加的元素的值
		\end{itemize}
\item reserve保留一定行的数量
	\begin{lstlisting}[language=c++]
	Mat::reserve(size_t sz)
	\end{lstlisting}
		\begin{itemize}
		\item sz \quad 行数
		\end{itemize}
\item push\_back将元素添加到矩阵底部
	\begin{lstlisting}[language=c++]
	Mat::push_back(const Mat& elem)
	\end{lstlisting}
		\begin{itemize}
		\item elem增加的一个或多个元素
		\end{itemize}
\item pop\_back从底部的列表删除元素
	\begin{lstlisting}[language=c++]
	Mat::pop_back(size_t nelems=1)
	\end{lstlisting}
		\begin{itemize}
		\item nelems删除的行的数目
		\end{itemize}
\end{itemize}
赋值关系
\begin{itemize}
\item clone
	\begin{lstlisting}[language=c++]
	cv::Mat F = M.clone(); 
	\end{lstlisting}
\item copyTo
	\begin{lstlisting}[language=c++]
	cv::Mat G; 
	F.copyTo(G); 
	\end{lstlisting}
\item convertTo
	\begin{lstlisting}[language=c++]
	convertTo ( OutputArray m , int rtype , double alpha =1, double beta =0 )
	\end{lstlisting}
	生成一个新矩阵,矩阵的中值为原矩阵中的值乘以alpha,然后再加上beta
\item assignTo
\item setTo
\end{itemize}



\section{class}

\subsection{KeyPoint}
Keypoint是opencv中的一个类,就是一个关键点,这个关键点存放了很多信息,同时也说明:想获得一个关键点,需要很多信息来初始化它。
\begin{itemize}
\item angle:角度,表示关键点的方向,通过Lowe大神的论文可以知道,为了保证方向不变形,SIFT算法通过对关键点周围邻域进行梯度运算,求得该点方向。-1为初值。
\item class\_id:当要对图片进行分类时,我们可以用class\_id对每个特征点进行区分,未设定时为-1,需要靠自己设定
\item octave:代表是从金字塔哪一层提取的得到的数据。
\item pt:关键点点的坐标
\item response:响应程度,代表该点强壮大小。response代表着该关键点how good,更确切的说,是该点角点的程度。
\item size:该点直径的大小
\end{itemize}

注意一个问题:keypoint只是保存了opencv的sift库检测到的特征点的一些基本信息,也就上面所说的这些,但sift所提取出来的特征向量其实不是在这个里面,特征向量通过SiftDescriptorExtractor 提取,结果放在一个Mat的数据结构中。这个数据结构才真正保存了该特征点所对应的特征向量。具体见后文对SiftDescriptorExtractor 所生成的对象的详解。

\section{imgproc}
包含的内容如下:
\begin{itemize}
\item 图像滤波
\item 直方图
\item 结构分析和形状描述
\item 图像变换
\end{itemize}
\subsection{图像滤波}
\subsubsection{bilateralFilter双边滤波}
\begin{lstlisting}[language=c++]
void bilateralFilter(InputArray src, OutputArray dst, int d, double sigmaColor, double sigmaSpace, int borderType=BORDER_DEFAULT )
\end{lstlisting}

\subsubsection{blur均值滤波}
\begin{lstlisting}[language=c++]
void blur(InputArray src, OutputArray dst, Size ksize, Point anchor=Point(-1,-1), int borderType=BORDER_DEFAULT )
\end{lstlisting}
	\begin{itemize}
	\item ksize \quad size(3,3) $\cdots$
	\end{itemize}

\subsubsection{boxFilter方框滤波}
\begin{lstlisting}[language=c++]
void boxFilter(InputArray src, OutputArray dst, int ddepth, Size ksize, Point anchor=Point(-1,-1), bool normalize=true, int borderType=BORDER_DEFAULT )
\end{lstlisting}
	\begin{itemize}
	\item ddepth输出图像的深度,-1代表使用原图深度
	\item ksize内核的大小
	\end{itemize}

\subsubsection{GaussianBlur高斯滤波}
\begin{lstlisting}[language=c++]
GaussianBlur(InputArray src, OutputArray dst, Size ksize, double sigmaX, double sigmaY=0, int borderType=BORDER_DEFAULT )
\end{lstlisting}
	\begin{itemize}
	\item sigmaX表示高斯核函数在X方向的的标准偏差
	\item sigmaY表示高斯核函数在Y方向的的标准偏差
	\end{itemize}

\subsubsection{medianBlur中值滤波}
\begin{lstlisting}[language=c++]
void medianBlur(InputArray src, OutputArray dst, int ksize)
\end{lstlisting}

\subsubsection{filter2D滤波器掩码}
\begin{lstlisting}[language=c++]
void filter2D(InputArray src, OutputArray dst, int ddepth, InputArray kernel, Point anchor=Point(-1,-1), double delta=0, int borderType=BORDER_DEFAULT )
\end{lstlisting}
	\begin{itemize}
	\item kernel给定的掩膜
	\end{itemize}


\subsection{数学形态学}

\subsubsection{getStructuringElement获得不同结构元素}
\begin{lstlisting}[language=c++]
Mat getStructuringElement(int shape, Size ksize, Point anchor=Point(-1,-1))
\end{lstlisting}
\begin{itemize}
\item shape结构元素的形状:
	\begin{itemize}
	\item 矩形(包括线形)MORPH\_RECT
	\item 椭圆(包括圆形)MORPH\_ELLIPSE
	\item 十字形MORPH\_CROSS\\
	\end{itemize}
\item ksize结构元素的大小
\end{itemize}

\subsubsection{dilate膨胀}
\begin{lstlisting}[language=c++]
void dilate(InputArray src, OutputArray dst, InputArray kernel, Point anchor=Point(-1,-1), int iterations=1, int borderType=BORDER_CONSTANT, const Scalar& borderValue=morphologyDefaultBorderValue() )
\end{lstlisting}
\begin{itemize}
\item src输入图像,图像的通道可以是任意的,但是图像位深应该为以下几种:CV\_8U, CV\_16U, CV\_16S, CV\_32F,CV\_64F
\item dst输出图像
\item kernel输入结构元素
\end{itemize}

\subsubsection{erode膨胀}
\begin{lstlisting}[language=C++]
void erode(InputArray src, OutputArray dst, InputArray kernel, Point anchor=Point(-1,-1), int iterations=1, int borderType=BORDER_CONSTANT, const Scalar& borderValue=morphologyDefaultBorderValue() )
\end{lstlisting}

\subsubsection{morphologyEx形态学操作}
\begin{lstlisting}[language=c++]
void morphologyEx(InputArray src, OutputArray dst, int op, InputArray kernel, Point anchor=Point(-1,-1), int iterations=1, int borderType=BORDER_CONSTANT, const Scalar& borderValue=morphologyDefaultBorderValue() )
\end{lstlisting}
\begin{itemize}
\item src输入图像,Mat类的对象即可。图像位深应该为以下五种之一:CV\_8U, CV\_16U,CV\_16S, CV\_32F 或CV\_64F。
\item dst输出图像,函数的输出参数,需要和源图片有一样的尺寸和类型。
\item op表示形态学运算的类型,可以是如下之一的标识符:
	\begin{itemize}
	\item MORPH\_OPEN – 开运算(Opening operation)
	\item MORPH\_CLOSE – 闭运算(Closing operation)
	\item MORPH\_GRADIENT -形态学梯度(Morphological gradient)
	\item MORPH\_TOPHAT - “顶帽”(“Top hat”)
	\item MORPH\_BLACKHAT - “黑帽”(“Black hat“)
	\end{itemize}
\item kernel形态学运算的内核。若为NULL时,表示的是使用参考点位于中心3x3的核。我们一般使用函数 getStructuringElement配合这个参数的使用。
\end{itemize}




\subsection{几何变换}

\subsubsection{LinearPolar}
将图像映射到线性极坐标空间中。
\begin{lstlisting}[language=c++]
void cvLinearPolar(const CvArr* src, CvArr* dst, CvPoint2D32f center, double maxRadius, int flags=CV_INTER_LINEAR+CV_WARP_FILL_OUTLIERS )
\end{lstlisting}
\begin{itemize}
\item src输入图像。 
\item dst输出图像。 
\item center变换的中心,输出图像在这里最精确。 
\item maxRadius反幅度的尺度参数。 
\item flags插值方法和以下选择标志的结合
	\begin{itemize}
	\item CV\_WARP\_FILL\_OUTLIERS 填充输出图像所有像素,如果这些点有和外点对应的,则置零。
	\item CV\_WARP\_INVERSE\_MAP表示矩阵由输出图像到输入图像的逆变换,并且因此可以直接用于像素插值。否则,函数从map\_matrix中寻找逆变换。 
	\end{itemize} 
\end{itemize}

\subsubsection{LogPolar}
将图像映射到对数极坐标空间中。
\begin{lstlisting}[language=c++]
void cvLogPolar(const CvArr* src, CvArr* dst, CvPoint2D32f center, double M, int flags=CV_INTER_LINEAR+CV_WARP_FILL_OUTLIERS )
\end{lstlisting}
\begin{itemize}
\item src输入图像。 
\item dst输出图像。 
\item center变换的中心,输出图像在这里最精确。 
\item M幅度的尺度参数。 
\item flags插值方法和以下选择标志的结合
	\begin{itemize}
	\item CV\_WARP\_FILL\_OUTLIERS 填充输出图像所有像素,如果这些点有和外点对应的,则置零。
	\item CV\_WARP\_INVERSE\_MAP表示矩阵由输出图像到输入图像的逆变换,并且因此可以直接用于像素插值。否则,函数从map\_matrix中寻找逆变换。 
	\end{itemize} 
\end{itemize}

\subsubsection{Remap}
对图像进行普通几何变换
\begin{lstlisting}[language=c++]
void remap(InputArray src, OutputArray dst, InputArray map1, InputArray map2, int interpolation, int borderMode=BORDER_CONSTANT, const Scalar& borderValue=Scalar())
\end{lstlisting}
\begin{itemize}
\item src输入图像. 
\item dst输出图像. 
\item mapx为x坐标的映射 (32fC1 image). 
\item mapy为y坐标的映射 (32fC1 image). 
\item flags插值方法和以下开关选项的组合:
	\begin{itemize}
	\item CV\_WARP\_FILL\_OUTLIERS填充边界外的像素. 如果输出图像的部分象素落在变换后的边界外,那么它们的值设定为 fillval。
	\end{itemize}
\end{itemize}

\subsubsection{Resize}
改变图像大小
\begin{lstlisting}[language=c++]
void resize(InputArray src, OutputArray dst, Size dsize, double fx=0, double fy=0, int interpolation=INTER_LINEAR )
\end{lstlisting}
\begin{itemize}
\item src输入图像. 
\item dst输出图像. 
\item dsize输出图像的尺寸
\item interpolation插值方法:
	\begin{itemize}
	\item CV\_INTER\_NN最近邻插值,
	\item CV\_INTER\_LINEAR双线性插值 (缺省使用)
	\item CV\_INTER\_AREA使用象素关系重采样。当图像缩小时候,该方法可以避免波纹出现。当图像放大时,类似于 CV\_INTER\_NN 方法
	\item CV\_INTER\_CUBIC立方插值
	\end{itemize}
\end{itemize}


\subsection{各种形式图像转换}

\subsubsection{adaptiveThreshold自适应阈值}
\begin{lstlisting}[language=c++]
void adaptiveThreshold(InputArray src, OutputArray dst, double maxValue, int adaptiveMethod, int thresholdType, int blockSize, double C)
\end{lstlisting}
\begin{itemize}
\item src输入图像
\item dst输出图像
\item maxValue自适应阈值的最大值
\item adaptiveMethod自适应阈值算法使用:CV\_ADAPTIVE\_THRESH\_MEAN\_C(均值) 或 CV\_ADAPTIVE\_THRESH\_GAUSSIAN\_C(加权)
\item thresholdType取阈值类型:CV\_THRESH\_BINARY或CV\_THRESH\_BINARY\_INV
\item blockSize用来计算阈值的象素邻域大小: 3, 5, 7, ...
\item C对方法的有关参数,对方法 CV\_ADAPTIVE\_THRESH\_MEAN\_C 和 CV\_ADAPTIVE\_THRESH\_GAUSSIAN\_C, 它是一个从均值或加权均值提取的常数(见讨论), 尽管它可以是负数。
\end{itemize}

\subsubsection{cvtColor色彩转换}
\begin{lstlisting}[language=c++]
void cvtColor(InputArray src, OutputArray dst, int code, int dstCn=0 )
\end{lstlisting}
\begin{itemize}
\item code色彩空间转换代码\\
有如下转换:
	\begin{itemize}
	\item RGB$\leftrightarrow$GRAY:COLOR\_BGR2GRAY, COLOR\_RGB2GRAY, COLOR\_GRAY2BGR, COLOR\_GRAY2RGB
	\item RGB$\leftrightarrow$CIE:COLOR\_BGR2XYZ, COLOR\_RGB2XYZ, COLOR\_XYZ2BGR, COLOR\_XYZ2RGB
	\item RGB$\leftrightarrow$YCrCb JPEG (or YCC):COLOR\_BGR2YCrCb, COLOR\_RGB2YCrCb, COLOR\_YCrCb2BGR, COLOR\_YCrCb2RGB
	\item RGB$\leftrightarrow$HSV:COLOR\_BGR2HSV, COLOR\_RGB2HSV, COLOR\_HSV2BGR, COLOR\_HSV2RGB
	$\cdots$
	\end{itemize}
\item dstCn表示dst图像的波段数,这个值默认是0,它可以从参数code中推断。 
\end{itemize}

\subsubsection{threshold固定阈值二值化}
\begin{lstlisting}[language=c++]
double threshold(InputArray src, OutputArray dst, double thresh, double maxval, int type)
\end{lstlisting}
\begin{itemize}
\item thresh二值化阈值
\item maxval二值化的最大值
\item type运算方法:THRESH\_BINARY,THRESH\_BINARY\_INV,THRESH\_TRUNC,THRESH\_TOZERO,THRESH\_TOZERO\_INV
\end{itemize}

\subsubsection{watershed分水岭算法}
\begin{lstlisting}[language=c++]
void watershed(InputArray image, InputOutputArray markers)
\end{lstlisting}
\begin{itemize}
\item image输入图像为3路8位
\item markers输出
\end{itemize}


\subsection{结构分析和形状描述}
\subsubsection{findContours}
\begin{lstlisting}[language=c++]
void findContours(InputOutputArray image, OutputArrayOfArrays contours, OutputArray hierarchy, int mode, int method, Point offset=Point())
void findContours(InputOutputArray image, OutputArrayOfArrays contours, int mode, int method, Point offset=Point())
\end{lstlisting}
\begin{itemize}
\item image为二值单通道图像
\item contours输出轮廓坐标
\item hiararchy参数和轮廓个数相同,每个轮廓contours[i]对应4个hierarchy元素hierarchy[i][0] ~hierarchy[i][3],分别表示后一个轮廓、前一个轮廓、父轮廓、内嵌轮廓的索引编号,如果没有对应项,该值设置为负数
\item mode表示轮廓的检索模式
	\begin{itemize}
	\item CV\_RETR\_EXTERNAL表示只检测外轮廓
	\item CV\_RETR\_LIST检测的轮廓不建立等级关系
	\item CV\_RETR\_CCOMP建立两个等级的轮廓,上面的一层为外边界,里面的一层为内孔的边界信息。如果内孔内还有一个连通物体,这个物体的边界也在顶层。
	\item CV\_RETR\_TREE建立一个等级树结构的轮廓
	\end{itemize}
\item method为轮廓的近似办法
	\begin{itemize}
	\item CV\_CHAIN\_APPROX\_NONE存储所有的轮廓点,相邻的两个点的像素位置差不超过1,即max(abs(x1-x2),abs(y2-y1))==1
	\item CV\_CHAIN\_APPROX\_SIMPLE压缩水平方向,垂直方向,对角线方向的元素,只保留该方向的终点坐标,例如一个矩形轮廓只需4个点来保存轮廓信息
	\item CV\_CHAIN\_APPROX\_TC89\_L1,CV\_CHAIN\_APPROX\_TC89\_KCOS使用teh-Chinl chain 近似算法
	\end{itemize}
\end{itemize}



\subsection{绘图相关函数}
opencv.hpp

\subsubsection{Scalar}
\begin{itemize}
\item Scalar(a,b,c)
	\begin{itemize}
	\item a为蓝色
	\item b为绿色
	\item c为红色
	\end{itemize}
\end{itemize}

\subsubsection{Rectangle}
\begin{lstlisting}[language=c++]
rectangle(Mat& img, Point pt1, Point pt2, const Scalar&color, int thickness=1,int lineType=8, int shift=0)
\end{lstlisting}
	\begin{itemize}
	\item img为画矩形的对象
	\item pt1所画矩形左上角点坐标
	\item pt2所画矩形右下角点坐标
	\end{itemize}
\begin{lstlisting}[language=c++]
rectangle(Mat& img,Rect rec, const Scalar&color, int thickness=1, int lineType=8,int shift=0 )
\end{lstlisting}
	\begin{itemize}
	\item rec为确定矩形的另一种方式,给定左上角的坐标和长宽
	\item color指定矩形的颜色
	\item thickness线宽
	\item lingType边框线型
	\item shift坐标点的小数点位数
	\end{itemize}

\subsubsection{Line}
\begin{lstlisting}[language=c++]
line(Mat& img, Point pt1, Point pt2, const Scalar& color, int thickness=1, int lineType=8, int shift=0)
\end{lstlisting}
	\begin{itemize}
	\item img为要画的对象
	\item pt1线条的起点坐标
	\item pt2线条的终点坐标
	\end{itemize}

\subsubsection{Circle}
\begin{lstlisting}[language=c++]
circle(Mat&img, Point center, int radius, const Scalar&color,int thickness=1, int lineType=8, int shift=0)
\end{lstlisting}
	\begin{itemize}
	\item img为要画圆的对象
	\item center为圆的中心坐标
	\item radius为圆的半径
	\end{itemize}

\subsubsection{Ellipse}
\begin{lstlisting}[language=c++]
ellipse(Mat& img, Point center, Size axes, double angle, double startAngle, double endAngle, const Scalar& color, int thickness=1, int lineType=8, int shift=0)
\end{lstlisting}
	\begin{itemize}
	\item center椭圆中心的坐标
	\item axes为椭圆长短轴的一半
	\item startAngle椭圆弧起始的角度
	\item endAngle椭圆弧终止的角度
	\end{itemize}
\begin{lstlisting}[language=c++]
ellipse(Mat& img, constRotatedRect& box, const Scalar& color, int thickness=1, int lineType=8)
\end{lstlisting}
	\begin{itemize}
	\item box指定椭圆中心和旋转角度的信息,通过 RotatedRect 或 CvBox2D. 这表示椭圆画在旋转矩形上(矩形是不可见的,只是指定了一个框而已)
	\end{itemize}

\subsubsection{PutText}
\begin{lstlisting}[language=c++]
putText(Mat& img, const string& text, Point org, int fontFace, double fontScale, Scalar color, int thickness=1, int lineType=8, bool bottomLeftOrigin=false )
\end{lstlisting}
	\begin{itemize}
	\item text要显示的文字
	\item org文字在图像中左下角的坐标
	\item fontFace字体类型,可选择字体:FONT\_HERSHEY\_SIMPLEX, FONT\_HERSHEY\_PLAIN$\cdots$所有类型都可以配合 FONT\_HERSHEY\_ITALIC使用,产生斜体效果。
	\item fontScale字体的大小
	\end{itemize}

\subsubsection{drawContours}
\begin{lstlisting}[language=c++]
void drawContours(InputOutputArray image, InputArrayOfArrays contours, int contourIdx, const Scalar& color, int thickness=1, int lineType=LINE_8, InputArray hierarchy=noArray(), int maxLevel=INT_MAX, Point offset=Point() )
\end{lstlisting}
\begin{itemize}
\item image目标图像
\item contours输入轮廓,轮廓被存放在vector<vector<Point> >中
\item contourldx这个参数是描述怎样画轮廓。如果为负数,则画出全部轮廓。
\item color表示化轮廓的颜色。
\end{itemize}




\section{shape}

\subsection{Hausdorff}
\begin{lstlisting}[language=c++]
int main()
{
    cv::Ptr <cv::HausdorffDistanceExtractor> hausdorff_ptr = cv::createHausdorffDistanceExtractor();
    vector<Point> pt1s;
    vector<Point> pt2s;
    pt1s.push_back(Point(0, 0));
    pt1s.push_back(Point(0, 4));
    pt2s.push_back(Point(4, 6));
    pt2s.push_back(Point(4, 2));
    float distance = hausdorff_ptr->computeDistance(pt1s, pt2s);
    cout << distance << endl;
}
\end{lstlisting}



\section{object delection}
\subsection{matchTemplate}
在模板块和输入图像之间寻找匹配,获得匹配结果图像。
\begin{lstlisting}[language=c++]
void matchTemplate(InputArray image, InputArray templ, OutputArray result, int method)
\end{lstlisting}
\begin{itemize}
\item image要搜索的图像对象
\item templ需要匹配的模板
\item result比较结果,必须是单通单32位浮点数
\item method比较算法,共有六种方法如下:
	\begin{itemize}
	\item CV\_TM\_SQDIFF平方差匹配法,最好的匹配为0,值越大匹配越差
	\item CV\_TM\_SQDIFF\_NORMED归一化平方差匹配法
	\item CV\_TM\_CCORR相关匹配法,采用乘法操作,数值越大表明匹配越好
	\item CV\_TM\_CCORR\_NORMED数值越小越好
	\item CV\_TM\_CCOEFF相关系数匹配法,最好的匹配为1,-1表示最差的匹配
	\item CV\_TM\_CCOEFF\_NORMED归一化相关系数匹配法
	\end{itemize}
\end{itemize}
例子:
\begin{lstlisting}[language=c++]
int main(int argc, char* argv[])
{
    string path1="/Users/wangruchen/work/program/part-matching/pic/1.jpg";
    string path2="/Users/wangruchen/work/program/part-matching/pic/1-part.png";
    Mat srcIm=imread(path1);
    Mat partIm=imread(path2);
    Mat result;
    matchTemplate(srcIm, partIm, result, TM_SQDIFF);
    normalize(result, result, 0, 1, NORM_MINMAX);//归一化
    double minVal,maxVal;
    Point minPoi,maxPoi;
    minMaxLoc(result, & minVal, &maxVal, &minPoi, &maxPoi);//寻找矩阵result中最大值和最小值的位置
    rectangle(srcIm, minPoi, Point2f(minPoi.x+partIm.cols,minPoi.y+partIm.rows), Scalar(255,0,0));//在匹配位置画出矩形
    imshow("result", srcIm);
    waitKey(0);
    return 0;
}
\end{lstlisting}



\section{图像读取和保存}
\subsection{imread}
载入一副图片。
\begin{lstlisting}[language=c++]
Mat imread(const String& filename, int flags=IMREAD_COLOR )
\end{lstlisting}
\begin{itemize}
\item flags
	\begin{itemize}
	\item CV\_LOAD\_IMAGE\_ANYDEPTH转换为16位32位,或者8位
	\item CV\_LOAD\_IMAGE\_COLOR转换为彩色
	\item CV\_LOAD\_IMAGE\_GRAYSCALE转换为灰度图
	\item 大于0返回为3通道彩色图像
	\item 小于0返回为1通道灰度图像
	\item 小于0返回加载图(alpha通道)
	\end{itemize}
\item 支持一下格式:.bmp, .dib, .jpeg, .jpg, .jpe, .jp2, .png, .webp, .pbm, .pgm, .ppm, .sr, .ras, .tiff, .tif
\end{itemize}




\section{数据类型及位数总结}
表示了矩阵中元素的类型以及矩阵的通道个数,它是一系列的预定义的常量,其命名规则为CV\_(位数)+(数据类型)+(通道数)。具体的有以下值:
\subsection{Unsigned 8bits}
CvMat数据结构参数:CV\_8UC1,CV\_8UC2,CV\_8UC3,CV\_8UC4

\subsection{Signed 8bits}
CvMat数据结构参数:CV\_8SC1,CV\_8SC2,CV\_8SC3,CV\_8SC4

\subsection{Unsigned 16bits}
CvMat数据结构参数:CV\_16UC1,CV\_16UC2,CV\_16UC3,CV\_16UC4

\subsection{Signed 16bits}
CvMat数据结构参数:CV\_16SC1,CV\_16SC2,CV\_16SC3,CV\_16SC4

\subsection{Signed 32bits}
CvMat数据结构参数:CV\_32SC1,CV\_32SC2,CV\_32SC3,CV\_32SC4

\subsection{Float 32bits}
CvMat数据结构参数:CV\_32FC1,CV\_32FC2,CV\_32FC3,CV\_32FC4

\subsection{Double 64bits}
CvMat数据结构参数:CV\_64FC1,CV\_64FC2,CV\_64FC3,CV\_64FC4







%%---------------------------------------------------------------------
\end{document}
